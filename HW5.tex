\documentclass[11pt,notitlepage]{article}
% Mise en page
\usepackage[left=1.5cm, right=1.5cm, lines=45, top=0.8in, bottom=0.7in]{geometry}
\usepackage{fancyhdr}
\usepackage{float}
\pagestyle{fancy}
\usepackage[most,breakable]{tcolorbox}
\usepackage{pdfcol, xcolor}
\usepackage{makecell} % Required for splitting lines in table cells
\usepackage{tikz}
\usetikzlibrary{bayesnet}
\usetikzlibrary{arrows.meta, positioning}
\usepackage[linesnumbered,ruled,vlined]{algorithm2e}
\usepackage{graphicx}
\graphicspath{{./images/}}
\usepackage{dsfont}
\usepackage{amssymb, amsmath, mathtools}
\usepackage{bm}
\usepackage[breaklinks=true,colorlinks,linkcolor=magenta,urlcolor=magenta,citecolor=black]{hyperref}
\usepackage{xpatch}
\usepackage[shortlabels]{enumitem}

\definecolor{MBlue}{RGB}{0,39,76}
\newcommand{\MBlue}[1]{{\color{MBlue}#1}}

\allowdisplaybreaks

\newtcolorbox[auto counter]{exercise}[1][]{%
	colback=yellow!10,
	colframe=MBlue,
	coltitle=white,
	use color stack,
	enforce breakable,
	enhanced,
	fonttitle=\bfseries,
	before upper={\parindent15pt\noindent}, 
	title={\color{white} #1}
}

\lhead{
	\textbf{University of Michigan}
}
\rhead{
	\textbf{Fall 24}
}
\chead{
	\textbf{STATS 551}
}
\cfoot{}

\newcommand{\abs}[1]{\left\vert#1\right\vert}
\newcommand{\floor}[1]{\left\lfloor#1\right\rfloor}
\newcommand{\prob}[2][]{\mathbb{P}_{#1}\left(#2\right)}
\newcommand{\mean}[2][]{\mathbb{E}_{#1}\left[#2\right]}
\newcommand{\var}[1]{\mathbb{V}\text{ar}\left(#1\right)}
\newcommand{\cov}[1]{\mathbb{C}\text{ov}\left(#1\right)}
\newcommand{\sign}[1]{\text{sign}\left(#1\right)}
\newcommand{\inner}[2]{\left\langle #1,#2\right\rangle}
\newcommand{\norm}[1]{\left\lVert #1\right\rVert}
\newcommand{\corr}[1]{\text{corr}\left(#1\right)}
\newcommand{\Xv}{\mathbf{X}}
\newcommand{\Yv}{\mathbf{Y}}
\newcommand{\Hj}{H_{-j}}
\newcommand{\tr}[1]{\text{tr}\left[#1\right]}
\newcommand{\Id}{\mathbf{I}}
\newcommand{\ZeroM}{\mathbf{0}}
\newcommand{\hyp}{\mathcal{H}}
\DeclareMathOperator*{\argmax}{arg\,max}
\DeclareMathOperator*{\argmin}{arg\,min}

%===========================================================
\begin{document}
	\begin{center}
		\huge{\MBlue{\textbf{Homework 5}}}		
		\vskip20pt
		\Large{
			\textbf{name:} your name
        }
	\end{center}
    
    \begin{tcolorbox}[colframe=black, colback=white, boxrule=0.5mm, arc=2mm, left=3mm, right=3mm, top=3mm, bottom=3mm]
        \textbf{\color{red} INSTRUCTIONS:}
        \begin{itemize}
            \item Submit the solutions to all questions as a single PDF file on Gradescope.
            \vspace{-1mm}
            \item Please show all your work to receive full credit.
            \vspace{-1mm}
            \item All answers must be typeset preferably using LaTeX (the template is provided, so please use it). Start each exercise on a new page and write your answers inside the \texttt{exercise} environments provided.% PDFs generated using RMarkdown are also accepted.
            \vspace{-1mm}
            \item For all coding questions, or questions for which you rely on any computations please include the code and the outputs that reproduce your analyses in your PDF (you can use Ctrl+P on your Colab notebook, save as PDF and merge with your submission document). Don't forget to run all cells before submitting. \\
            If a question only involves coding, write ``Code provided below.'' in the solutions box and insert the PDF pages with the relevant code on the next page.
            \vspace{-1mm}
            \item When submitting on Gradescope, please indicate which pages of your solution are associated with each problem.
            \vspace{-1mm}
            \item You are free to discuss homework solutions with other students, but write-ups must be done independently. Be sure to give credit to any students you collaborate with. 
        \end{itemize}
        Failure to comply with these instructions will result in a deduction of points.
    \end{tcolorbox}
        

    \vskip10pt
    \noindent
    \textbf{\large \MBlue{Optional Questions}}
    \begin{enumerate}
        \item Do you have any confusion or questions about the previous lectures? 
        \item Any suggestions or thoughts about the course?
    \end{enumerate}
	\begin{exercise}[Suggestions]
    \end{exercise}

    \newpage
    \noindent
    \textbf{\large \MBlue{Exercise 1}}
    \vskip10pt
    \noindent
    \textbf{Traffic Study on Residential Streets [30 points].} A survey was done of bicycle and other vehicular traffic in the neighborhood of the campus of the University of California, Berkeley, in the spring of 1993. Sixty city blocks were selected at random; each block was observed for one hour,
    and the numbers of bicycles and other vehicles traveling along that block were recorded. The sampling was stratified into six types of city blocks: busy, fairly busy, and residential streets, with and without bike routes, with ten blocks measured in each stratum. The following table displays the number of bicycles and other vehicles recorded in the study. 
    \begin{center}
        \begin{tabular}{llp{8cm}}
            \hline
            \textbf{Type of street} & \textbf{Bike route?} & \textbf{Counts of bicycles/other vehicles} \\
            \hline
            Residential & yes & 16/58, 9/90, 10/48, 13/57, 19/103, 20/57, 18/86, 17/112, 35/273, 55/64 \\
            Residential & no & 12/113, 1/18, 2/14, 4/44, 9/208, 7/67, 9/29, 8/154 \\
            Fairly busy & yes & 8/29, 35/415, 31/425, 19/42, 38/180, 47/675, 44/620, 44/437, 29/47, 18/462 \\
            Fairly busy & no & 10/557, 43/1258, 5/499, 14/601, 58/1163, 15/700, 0/90, 47/1093, 51/1459, 32/1086 \\
            Busy & yes & 60/1545, 51/1499, 58/1598, 59/503, 53/407, 68/1494, 68/1558, 60/1706, 71/476, 63/752 \\
            Busy & no & 8/1248, 9/1246, 6/1596, 9/1765, 19/1290, 61/2498, 31/2346, 75/3101, 14/1918, 25/2318 \\
            \hline
        \end{tabular}
    \end{center}
    Now restrict, your attention to the first four rows of the table: the data on residential streets.
    \begin{enumerate}
        \item \textbf{[8 points]} Let $y_1,\ldots, y_{10}$ and $z_1,\ldots, z_8$ be the observed proportion of traffic that was on bicycles in the residential streets with bike lanes and with no bike lanes, respectively (so $y_1 = 16/(16+58)$ and $z_1=12/(12+13)$ for example). Set up a model so that the $y_i$'s are independent and identically distributed given parameters $\theta_y$ and the $z_i$'s are independent and identically distributed given parameters $\theta_z$.
        \item \textbf{[4 points]} Set up a prior distribution that is independent in $\theta_y$ and $\theta_z$.
        \item \textbf{[10 points]} Draw 1000 simulations from the posterior distribution. (Hint: $\theta_y$ and $\theta_z$ are independent in the
        posterior distribution, so they can be simulated independently.) \label{1c}
        \item \textbf{[8 points]} Let $\mu_y = E(y_i|\theta_y)$ be the mean of the distribution of the $y_i$'s; $\mu_y$ will be a function of $\theta_y$. Similarly, define $\mu_z$. Using your posterior simulations from~\ref{1c}, plot a histogram of the posterior simulations of $\mu_y-\mu_z$, the expected difference in proportions in bicycle traffic on residential streets with and without bike lanes.
    \end{enumerate}
	\begin{exercise}[Solution]
        
    \end{exercise}


    \newpage
    \noindent
    \textbf{\large \MBlue{Exercise 2}}
    \vskip10pt
    \noindent
    \textbf{Regression with many explanatory variables [40 points]}. Table 15.2 displays data from a designed experiment for a chemical process. In using these data to illustrate various approaches
    to selection and estimation of regression coefficients, Marquardt and Snee (1975) assume
    a quadratic regression form; that is, a linear relation between the expectation of
    the untransformed outcome, y, and the variables $x_1,x_2,x_3$, their two-way interactions,
    $x_1x_2, x_1x_3, x_2x_3$, and their squares, $x_1^2, x_2^2, x_3^2$.
    \begin{table}[h!]
        \centering
        \caption{Data from a chemical experiment, from Marquardt and Snee (1975). The first three variables are experimental manipulations, and the fourth is the outcome measurement.}
        \begin{tabular}{cccc}
        \hline
        \makecell{Reactor temperature \\ (\(^\circ\text{C}\)), \(x_1\)} & 
        \makecell{Ratio of H\(_2\) to \\ \(n\)-heptane (mole ratio), \(x_2\)} & 
        \makecell{Contact time \\ (sec), \(x_3\)} & 
        \makecell{Conversion of \\ \(n\)-heptane to acetylene (\%), \(y\)} \\
        \hline
        1300 & 7.5  & 0.0120 & 49.0 \\
        1300 & 9.0  & 0.0120 & 50.2 \\
        1300 & 11.0 & 0.0115 & 50.5 \\
        1300 & 13.5 & 0.0130 & 48.5 \\
        1300 & 17.0 & 0.0135 & 47.5 \\
        1300 & 23.0 & 0.0120 & 44.5 \\
        1200 & 5.3  & 0.0400 & 28.0 \\
        1200 & 7.5  & 0.0380 & 31.5 \\
        1200 & 11.0 & 0.0320 & 34.5 \\
        1200 & 13.5 & 0.0260 & 35.0 \\
        1200 & 17.0 & 0.0340 & 38.0 \\
        1200 & 23.0 & 0.0410 & 38.5 \\
        1100 & 5.3  & 0.0840 & 15.0 \\
        1100 & 7.5  & 0.0980 & 17.0 \\
        1100 & 11.0 & 0.0920 & 20.5 \\
        1100 & 17.0 & 0.0860 & 19.5 \\
        \hline
        \end{tabular}
    \end{table}

    Data is available here: \url{http://www.stat.columbia.edu/~gelman/book/data/factorial.asc}
    \begin{enumerate}
        \item \textbf{[5 points]} Fit an ordinary linear regression model (OLS frequentist model), including a constant term and the nine explanatory
        variables above.\label{a}
        \item \textbf{[10 points]} Fit a Bayesian linear regression model with a uniform prior distribution on the
        constant term and a shared normal prior distribution on the coefficients of the nine
        variables above. If you use iterative simulation in your computations, be sure to use
        multiple sequences and monitor their joint convergence.\label{b}
        \item \textbf{[10 points]} Discuss the differences between the inferences in (\ref{a}) and (\ref{b}). Interpret the differences in terms of the hierarchical variance parameter. Do you agree with Marquardt and
        Snee that the inferences from (\ref{a}) are unacceptable?
        \item \textbf{[8 points]} Repeat (\ref{a}), but with a $t_4$ prior distribution on the nine variables.
        \item \textbf{[7 points]} Discuss other models for the regression coefficients.
    \end{enumerate}
    \begin{exercise}[Solution]
    
    \end{exercise}
    

    \newpage
    \noindent
    \textbf{\large \MBlue{Exercise 3}}
    \vskip10pt
    \noindent
    \textbf{Hierarchical model [30 points].} This is a question that intends to help you with your final project. Find a data set on Kaggle or UCI  Machine Learning Repository (you can reuse the dataset you chose from the previous homework or choose a new one), is there any group structures in the dataset? Can youå expand your one-parameter model in the previous homeworks into a hierarchical model? Interpret the results from both model fits; what do they tell you about the dataset?
    \vskip10pt\noindent
    Compare the one-parameter model and the hierarchical model fit; consider performing posterior predictive check. Which model fit the data better?
    \vskip10pt\noindent
    This question will be graded based on effort, as opposed to correctness. You will get full mark if substantial effort is demonstrated.
    \begin{exercise}[Solution]
    
    \end{exercise}
    
\end{document}